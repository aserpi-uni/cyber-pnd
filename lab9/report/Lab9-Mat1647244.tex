\documentclass[draft]{homework}

\usepackage{fixme}
\usepackage{graphicx}


\title{Practical Network Defense - Lab 9}
\subtitle{Vulnerability assessment of ACME co.'s network}
\author{Alessandro Serpi - 1647244}
\date{17 May 2019}


\begin{document}
    \maketitle
    \tableofcontents
    
    
    \pagebreak
    \section{Introduction}
    \fxnote{TODO}
    
    
    \section{Greenbone setup}
    \fxnote{TODO}
    
    \subsection{Tasks}
    Login in the web GUI and create a new host in \textit{Assets} $\triangleright$ \textit{Hosts} for every machine in the network, filling \textit{Name} with the host's IP address and \textit{Comment} with a description.
    \vspace{-5pt}
    \begin{figure}[H]
        \centering
        \includegraphics[width=1\linewidth]{images/new-host}
        \label{fig:new-host}
    \end{figure}
    \vspace{-20pt}
    
    Next, create new credentials in \textit{Configuration} $\triangleright$ \textit{Credentials} for every used SSH username/password combination.
    \vspace{-5pt}
    \begin{figure}[H]
        \centering
        \includegraphics[width=1\linewidth]{images/new-credentials}
        \label{fig:new-credentials}
    \end{figure}
    \vspace{-20pt}
    
    In the host menu, create for each host a new target using the specific button.
    In the configuration window, select \textit{All IANA assigned TCP and UDP 2012-02-20} as \textit{Port List} and add the SSH credentials set (if it exists).
    
    Finally, in \textit{Scans} $\triangleright$ \textit{Tasks}, create a new task for every target, selecting \textit{Full and fast ultimate} as \textit{Scan Config} and \textit{Random} as \textit{Order for target hosts}.
    
    
    \section{Rules of engagement}
    \fxnote{TODO}
    
    
    \section{Assessment results and analysis}
    \subsection{Missing cookie attributes in the domain controller}
    \begin{displayquote}
        The application is missing the `httpOnly' attribute.
        \textelp{}
        This allows a cookie to be accessed by JavaScript which could lead to session hijacking attacks.
    \end{displayquote}
    \begin{displayquote}
        The flaw is due to cookie is not using 'secure' attribute, which allows cookie to be passed to the server by the client over non-secure channels (http) and allows attacker to conduct session hijacking attacks.
    \end{displayquote}
    The domain controller does not set `httpOnly' and `secure' cookie attributes, which can allow session hijacking attacks.
    Since this is a Zentyal's shortcoming, we can not implement any mitigation.
    
    \subsection{SSH brute force login with default credentials}
    \begin{displayquote}
        It was possible to login into the remote SSH server using default credentials.
       \textelp{}
       Change the password as soon as possible.
    \end{displayquote}
    In the remote environment, we left the default password for the majority of the resources.
    There were no such vulnerabilities in the local environment: when we created the servers, we chose non-default username/password combinations.
    
    \subsection{TCP timestamps}
    \begin{displayquote}
        It was detected that the host implements RFC1323.
        \textelp{}
        A side effect of this feature is that the uptime of the remote host can sometimes be computed.
    \end{displayquote}
    
    An attacker may use TCP timestamps to determine whether security patches requiring a reboot were applied to a host.
    However, disabling timestamps adds security through obscurity, which is no security. 
    In addition, TCP timestamps are used in the PAWS (Protect Against Wrapped Sequence Numbers) mechanism.
    
    Therefore, we decided not to consider the user of this functionality a vulnerability and not to implement any mitigation.
    
    \fxnote{TODO}
    
    
    \section{Mitigations}
    \subsection{SSH brute force login with default credentials}
    Use non-predictable passwords also in the remote environment.
    
    \fxnote{TODO}
    
    
    \section{Final remarks}
    \fxnote{TODO}
\end{document}
