\documentclass[draft]{homework}

\usepackage{fixme}


\title{Practical Network Defense - Lab 9}
\subtitle{Vulnerability assessment of ACME co.'s network}
\author{Alessandro Serpi - 1647244}
\date{17 May 2019}


\begin{document}
    \maketitle
    \tableofcontents
    
    
    \pagebreak
    \section{Introduction}
    \fxnote{TODO}
    
    
    \section{Greenbone setup}
    \fxnote{TODO}
    
    
    \section{Rules of engagement}
    \fxnote{TODO}
    
    
    \section{Assessment results and analysis}
    \subsection{TCP timestamps}
    \begin{displayquote}
        It was detected that the host implements RFC1323.
        \textelp{}
        A side effect of this feature is that the uptime of the remote host can sometimes be computed.
    \end{displayquote}
    
    An attacker may use TCP timestamps to determine whether security patches requiring a reboot were applied to a host.
    However, disabling timestamps adds security through obscurity, which is no security. 
    In addition, TCP timestamps are used in the PAWS (Protect Against Wrapped Sequence Numbers) mechanism.
    
    Therefore, we decided not to consider the user of this functionality a vulnerability and not to implement any mitigation.
    
    \fxnote{TODO}
    
    
    \section{Mitigations}
    \fxnote{TODO}
    
    
    \section{Final remarks}
    \fxnote{TODO}
\end{document}
