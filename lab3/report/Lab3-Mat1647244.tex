\documentclass[draft]{homework}

\usepackage{graphicx}
\usepackage{xspace}


\newcommand{\kat}{Kathará\xspace}

\newcommand{\client}{\textit{client}\xspace}
\newcommand{\dmz}{\textit{DMZ}\xspace}
\newcommand{\ser}{\textit{internal server}\xspace}
\newcommand{\intfw}{\textit{intfw}\xspace}
\newcommand{\mainfw}{\textit{mainfw}\xspace}


\title{Assignment 3: iptables on ACME Co.}
\author{Alessandro Serpi - 1647244}
\date{8 March 2019}


\begin{document}
    \maketitle
    \tableofcontents
    
    \pagebreak
    \section{Introduction}
    The private network of \textit{ACME co.} is composed by four subnetworks living in the shared address space \texttt{100.64.0.0/16}. \dmz is connected directly to the main firewall-router \mainfw and offers services accessible from the external network, while \client and \ser networks are behind a second line of defence (represented by the internal firewall-router \intfw). The latter offers services that may be used only by hosts in the private network, whereas the former does not offer any services.
    
    All hosts except the firewalls are already configured. 
    
    
    \section{Evaluation of the security policy}
    \fxnote{TODO}
    
    
    \section{Policy implementation}
    \fxnote{TODO}
    
    
    \section{Test of the configuration}
    We use the test environment integrated in \kat.
    To verify the correctness of the implementation, execute \texttt{ltest --verify=user} in the lab root folder.
    For each host, faulty elements are listed in the \texttt{\_test/diff} folder.
    Note that there may be some false positive.
    
    Given that \textit{expect} is not installed and both \textit{ftp} and \textit{ssh} clients can not be reliably used with simple bash scripts, testing the corresponding protocols must be done manually, checking the connectivity for each host.
    
    
    \section{Final remarks}
    \fxnote{TODO}
\end{document}