\documentclass[draft]{homework}

\usepackage{fixme}
\usepackage{minted}
\usepackage{xspace}

\newcommand{\kat}{Kathará\xspace}
\newcommand{\sq}{Squid\xspace}


\title{Practical Network Defense - Lab 8}
\subtitle{\sq on ACME co.}
\author{Alessandro Serpi - 1647244}
\date{10 May 2019}


\begin{document}
    \maketitle
    \tableofcontents
    
    
    \pagebreak
    \section{Introduction}
   \fxnote{TODO}
    
    
    \section{\sq installation}
    \fxnote{TODO}
    
    
    \section{Forward proxy configuration}
    \fxnote{TODO}
    
    
    \section{Authentication setup}
    \fxnote{TODO}
    
    
    \section{Test of the setup}
    The IP address \texttt{100.64.2.5/24} (belonging to the client network) was assigned to a virtual interface of the host and the route \texttt{100.64.0.0/16} was added to the same interface via \texttt{100.64.2.1} (the internal firewall's address in the client network).
    
    A \kat guest was created in the client network.
    When it tried to connect to a web server (either in the private network or in the internet) trough \textit{links}, no page could be reached.
    
    In the host OS, a Mozilla Firefox instance was configured to use \texttt{100.64.6.3} as web proxy.
    When it tried to connect to a web server, the proxy requested user credentials.
    It was possible to navigate only if the inserted username and password corresponded to a valid LDAP user account.
    
    
    \section{Final remarks}
    \fxnote{TODO}
\end{document}
