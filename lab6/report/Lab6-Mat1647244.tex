\documentclass[draft]{homework}

\usepackage{xspace}


\newcommand{\kat}{Kathará\xspace}
\newcommand{\opn}{OPNsense\xspace}
\newcommand{\vb}{VirtualBox\xspace}

\newcommand{\client}{\textit{client}\xspace}
\newcommand{\dmz}{\textit{DMZ}\xspace}
\newcommand{\ser}{\textit{internal server}\xspace}
\newcommand{\intfw}{\textit{intfw}\xspace}
\newcommand{\mainfw}{\textit{mainfw}\xspace}

\newcommand{\lan}{\textit{LAN}\xspace}
\newcommand{\opt}{\textit{OPT1}\xspace}
\newcommand{\wan}{\textit{WAN}\xspace}


\title{Practical Network Defense - Lab 6}
\author{Alessandro Serpi - 1647244}
\date{3 May 2019}


\begin{document}
    \maketitle
    \tableofcontents
    
    
    \pagebreak
    \section{Introduction}
    \fxnote{TODO}
    
    
    \section{Zentyal configuration}
    \fxnote{TODO}
    
    
    \section{\opn internal firewall configuration}
    \fxnote{TODO}
    
    
    \section{\opn main firewall configuration}
    \fxnote{TODO}
    
    
    \section{Test of the configuration}
    Testing was performed manually, trying to login in the two firewalls, both in the web GUI and the physical terminal.
    Using the built-in root user had a negative outcome, while using a LDAP account gave access to both the web GUI and the shell.
    
    
    \section{Final remarks}
    \fxnote{TODO}
\end{document}
